%
% File coling2020.tex
%
% Contact: feiliu@cs.ucf.edu & liang.huang.sh@gmail.com
%% Based on the style files for COLING-2018, which were, in turn,
%% Based on the style files for COLING-2016, which were, in turn,
%% Based on the style files for COLING-2014, which were, in turn,
%% Based on the style files for ACL-2014, which were, in turn,
%% Based on the style files for ACL-2013, which were, in turn,
%% Based on the style files for ACL-2012, which were, in turn,
%% based on the style files for ACL-2011, which were, in turn, 
%% based on the style files for ACL-2010, which were, in turn, 
%% based on the style files for ACL-IJCNLP-2009, which were, in turn,
%% based on the style files for EACL-2009 and IJCNLP-2008...

%% Based on the style files for EACL 2006 by 
%%e.agirre@ehu.es or Sergi.Balari@uab.es
%% and that of ACL 08 by Joakim Nivre and Noah Smith

\documentclass[11pt]{article}
\usepackage{coling2020}
\usepackage{times}
\usepackage{url}
\usepackage{latexsym}



%\setlength\titlebox{5cm}

% You can expand the titlebox if you need extra space
% to show all the authors. Please do not make the titlebox
% smaller than 5cm (the original size); we will check this
% in the camera-ready version and ask you to change it back.


\title{DismemBERT: Detecting Diachronic Lexical Semantic Change Using BERT Embeddings in an Unsupervised Knowledge-Free Setting}

\author{David Rother \\
  TU Darmstadt \\
  {\tt david.rother@stud.tu-darmstadt.de} \\\And
  Thomas Haider \\
  {\tt thomas.haider@ae.mpg.de} \\\And
  Steffen Eger \\
  {\tt eger@aiphes.tu-darmstadt.de} \\}

\date{}

\begin{document}
\maketitle
\begin{abstract}
  This document contains the instructions for preparing a paper submitted
  to COLING-2020 or accepted for publication in its proceedings. The document itself
  conforms to its own specifications, and is therefore an example of
  what your manuscript should look like. These instructions should be
  used for both papers submitted for review and for final versions of
  accepted papers. Authors are asked to conform to all the directions
  reported in this document.
\end{abstract}

\section{Introduction}
Here is something regarding semanitc change \cite{schlechtweg2018diachronic}

\section{Related Work}

\subsection{Diachronic Lexical Semantic Change}
With an increasing interest in Diachronic Lexical Semantic Change (LSC)  
there is a multitude of approaches and 
three different word representations are commonly used \cite{schlechtweg2019wind}. \newline
First are semantic vector representations such as word2vec \cite{mikolov2013efficient},
which represents each word with two different vectors for each time period respectively \cite{hamilton2016cultural,hamilton2016diachronic}.
The vectors itself represent the co-occurence statistics of the word in the given time period. \newline
Second is the use ditributional representations of words. To this end
\cite{frermann2016bayesian} use bayesian learning. \newline
Third is 

\subsection{Word Sense Disambiguation}

\subsection{Unsupervised Knowledge-free Sense Modelling}


\section{Corpora}

\begin{table}[h!]
  \label{corpora_epochs}
  \begin{tabular}{|l|l|l|}
  \hline
          & t1        & t2        \\ \hline
  English & 1810-1860 & 1960-2010 \\ \hline
  German  & 1810-1860 & 1945-1990 \\ \hline
  Swedish & 1800-1830 & 1900-1925 \\ \hline
  Latin   & -200-0    & 0-2000    \\ \hline
  \end{tabular}
\end{table}

The Corpora for evaluation are from the SEMEVAL 2020 Task 1: "Unsupervised Lexical Semantic Change Detection".
They contain lemmatized text for english, german, swedish and latin. For each language two corpora are available from two distinct time periods.
The respective Time periods can be seen in ref to table.

\section{Framework}

\section{Experiments}
\section{Evaluation} 
Hi
\section{Conclusion}

% include your own bib file like this:
\bibliographystyle{acl}
\bibliography{coling2020}

\end{document}
