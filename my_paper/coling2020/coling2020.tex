%
% File coling2020.tex
%
% Contact: feiliu@cs.ucf.edu & liang.huang.sh@gmail.com
%% Based on the style files for COLING-2018, which were, in turn,
%% Based on the style files for COLING-2016, which were, in turn,
%% Based on the style files for COLING-2014, which were, in turn,
%% Based on the style files for ACL-2014, which were, in turn,
%% Based on the style files for ACL-2013, which were, in turn,
%% Based on the style files for ACL-2012, which were, in turn,
%% based on the style files for ACL-2011, which were, in turn, 
%% based on the style files for ACL-2010, which were, in turn, 
%% based on the style files for ACL-IJCNLP-2009, which were, in turn,
%% based on the style files for EACL-2009 and IJCNLP-2008...

%% Based on the style files for EACL 2006 by 
%%e.agirre@ehu.es or Sergi.Balari@uab.es
%% and that of ACL 08 by Joakim Nivre and Noah Smith

\documentclass[11pt]{article}
\usepackage{coling2020}
\usepackage{times}
\usepackage{url}
\usepackage{latexsym}



%\setlength\titlebox{5cm}

% You can expand the titlebox if you need extra space
% to show all the authors. Please do not make the titlebox
% smaller than 5cm (the original size); we will check this
% in the camera-ready version and ask you to change it back.


\title{DismemBERT: Detecting Diachronic Lexical Semantic Change Using BERT Embeddings}

\author{David Rother \\
  TU Darmstadt \\
  {\tt david.rother@stud.tu-darmstadt.de} \\}
\date{}

\begin{document}
\maketitle
\begin{abstract}
  This document contains the instructions for preparing a paper submitted
  to COLING-2020 or accepted for publication in its proceedings. The document itself
  conforms to its own specifications, and is therefore an example of
  what your manuscript should look like. These instructions should be
  used for both papers submitted for review and for final versions of
  accepted papers. Authors are asked to conform to all the directions
  reported in this document.
\end{abstract}

\section{Introduction}
Here is something regarding semanitc change \cite{schlechtweg2018diachronic}

\section{Related Work}

\subsection{Diachronic Lexical Semantic Change}
With an increasing interest in Diachronic Lexical Semantic Change (LSC)  
there is a multitude of approaches and 
three different word representations are commonly used \cite{schlechtweg2019wind}. \newline
First are semantic vector representations such as word2vec \cite{mikolov2013efficient},
which represents each word with two different vectors for each time period respectively \cite{hamilton2016cultural,hamilton2016diachronic}.
The vectors itself represent the co-occurence statistics of the word in the given time period. \newline
Second is the use ditributional representations of words. 
A word is represented as a vector over all other words occuring in its context.
The actual distribution can then be obtained by the word-context co-occurence matrix.
To learn these distributions \cite{frermann2016bayesian} use bayesian learning. \newline
Third are sense clusters where each occurence of a word is assigned to a sense cluster.
The clustering usually happens according to some contextual property \cite{mitra2015automatic}.
In newer approaches powerful pretrained deep neural networks such as BERT \cite{hu2019diachronic,devlin2018bert}
are used to extract directly a contextual token of a word from a sentence.

\subsection{Word Sense Disambiguation}

Word Sense Disambiguation (WSD) is the task of finding different word senses of the same word in sentences.
Supervised WSD has sense annotated data and a system usually directly tries to learn
Sense Embeddings. 
The major disadvantage using this approach is that annotating data is very expensive and usually a sufficient amount of data can not be provided. \newline
Unsupervised learning on the other side does not suffer from such constraints. There either leverages 
some kind of knowledge base such as BabelNet or WordNet, or use a knowledge free model that induces senses \cite{panchenko2017unsupervised}.

\section{Corpora}

\begin{table}[h!]
  \label{corpora_epochs}
  \begin{tabular}{|l|l|l|}
  \hline
          & t1        & t2        \\ \hline
  English & 1810-1860 & 1960-2010 \\ \hline
  German  & 1810-1860 & 1945-1990 \\ \hline
  Swedish & 1800-1830 & 1900-1925 \\ \hline
  Latin   & -200-0    & 0-2000    \\ \hline
  \end{tabular}
\end{table}

The Corpora for evaluation are from the SEMEVAL 2020 Task 1: "Unsupervised Lexical Semantic Change Detection".
They contain lemmatized text for english, german, swedish and latin. For each language two corpora are available from two distinct time periods.
The respective Time periods can be seen in ref to table. The english corpus is a cleaned version of the COHA corpus \cite{davies2002corpus}, 
where the corpus has been transformed and every tenth word is replaced by an @. The organizers split the sentences at these tokens and removed them.
The german corpus uses the DTA corpus, the BZ ,and the ND corpus. 

\section{Framework}

In this section we present our framework to solve the SemEval 2020 Task 1. 
We select to apply the same pipeline to all four languages. For each language we have two corpora
given and a list of words to compute the LSC on. Since no further fine tuning is done on the provided 
corpus data we avoid having to align the resulting embedding spaces and it can even be shown that fine tuning may be decreasing
with small corpora \cite{giulianelli2019lexical} probably due to overfitting. 

\subsection{Contextualized Embeddings}

We start by computing the contextualized embeddings for each word. To that end we use 
the Huggingface implementation \cite{Wolf2019HuggingFacesTS}. For the english corpus we compute the embeddings using the
bert-base-cased model. The german embeddings are computed with the bert-base-german-cased
pretrained model and swedish and latin embeddings are computed using the bert-base-multilingual-cased model that uses the 104 languages with the largest wikipedias.

\subsection{Preprocessing}

To be able to do efficient clustering on the contextualized embeddings we use a preprocessing pipeline to
enhance later results. 
As a starting point \cite{reif2019visualizing} show that BERT embeddings projected with UMAP \cite{mcinnes2018umap}, a type of manifold learning similar to t-SNE,
produces distinct clusters of the different word senses. 
Furthermore, \cite{mcconville2019n2d} show that using an autoencoder in conjunction with UMAP
leads to higher quality clusterings and that their approach is competetive with other unsupervised deep learning clustering methods.
We decide to adopt the latter pipeline and choose a similar autoencoder network with the only change being that we fix the dimension of the latent representation space to be 20 for all words
since we have no knowledge of the true amount of senses beforehand.


\section{Experiments}

In the SemEval-2020 Task 1 challenge there are two different sub-tasks to solve. 
In the first task one has to decide wether a word has either gained or lost a sense or 
if the senses remained the same between two time periods. And in the second task the model 
has to rank words based on the magnitude of change they did undergo. 
The task organizers quantified the amount of semantic change by constructing a sense frequency distribution
for both epochs by human experts. The change score is then the jensen-shannon frequency divergence of the two resulting distributions.

\section{Evaluation} 
The score is ok.

\section{Conclusion}
This needs work.

% include your own bib file like this:
\bibliographystyle{acl}
\bibliography{coling2020}

\end{document}
